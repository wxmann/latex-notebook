\documentclass[11pt]{article}
\usepackage{geometry}
\usepackage{graphicx}
\usepackage{amsmath}
\usepackage{hyperref}
\usepackage{sectsty}
\usepackage{cancel}
\usepackage{float}
\usepackage{placeins} 
\usepackage{enumitem}
%\usepackage{chngcntr}
\sectionfont{\large}
\subsectionfont{\normalsize}
\setlength{\parindent}{0cm}
%\setlength{\parskip}{10pt}
%\counterwithin*{equation}{section}
\geometry{top=3cm,bottom=3cm,left=3cm,right=3cm}
\setlist{nosep}

\title{How often should you fill up on gas?}
\author{Jim Tang}
\begin{document}
  \maketitle

Over the past two weeks, I have observed an enormous increase in gas prices. Unfortunately, having not filled up on gas in awhile, I missed it when the prices bottomed out. Perhaps, if I had filled up on gas more often, I may have been able to get it when it was at its lowest. This inspired me to ask the question: in general, is it better to fill up more or less frequently, in smaller or bigger chunks?

\section*{Calculation}
%\addtocounter{section}{1}
\label{sec:calculation}

Gas prices are known to fluctuate back-and-forth. Mathematically, we can model this with a sinusoidal function. (Recall that any oscillatory signal can be expressed us a sum of sinusoids, so this is a good starting point.) Let $G(t)$ be the gas price as a function of time. With this model,
\begin{equation}
G(t) = G_0 + G_a \sin \left( \frac{2 \pi t}{f}\right) \label{eq:gas_prices},
\end{equation}

where $G_0$ is the average gas price, $G_a$ is the maximum oscillation from the average, and $f$ is the period of oscillation. The values of $G_a / G_0$ and $f$ determine the volatility of prices.

\par Let's say we fill up on gas at constant intervals of time $\tau$, and we do this for a total time $T$, such that during the total time we fill up $N$ times, where
 $$N = \frac{T}{\tau}.$$ 

Every time, we want to fill up our vehicle until it is full, which will require $b \tau$ amount of gas, where $b$ is some proportionality constant with dimensions of [amount of gas] / [time]. The cost of filling up gas the $k$th time $C_k$ is the amount of gas multiplied by the cost of gas at time $t = k \tau$, or
\begin{equation}
C_k = b\tau \, G(t = k \tau) = b\tau \left( G_0 + G_a \sin \left( \frac{2 \pi k \tau}{f}\right) \right).
\end{equation}

The total cost of gas through time $T$, which we will denote as $C$, is
\begin{align}
C = \sum_{k=1}^{N}{C_k} &= \sum_{k=1}^{N}{ b\tau \left( G_0 + G_a \sin \left( \frac{2 \pi k \tau}{f}\right) \right)} \nonumber \\ 
 &= \underbrace{\sum_{k=1}^{N}{ b\tau G_0}}_{=Nb \tau G_0} + \: b \tau G_a \sum_{k=1}^{N} \sin \left( \frac{2 \pi k \tau}{f}\right). \label{eq:sum_C_k}
\end{align}

Define 
$$
C_0 \equiv Nb\tau G_0 = TbG_0
$$

 as the base cost, the cost of gas if you hit the average cost every time you filled up gas. Moving all terms without $\tau$ to the LHS yields
\begin{equation}
\frac{C - C_0}{b G_a} = \tau \sum_{k=1}^{N}{\sin \left( \frac{2 \pi k \tau}{f}\right)} \label{eq:result_dimensional}.
\end{equation}

Nondimensionalizing the equation, we define a dimensionless time $q$
$$
q \equiv \frac{\tau}{f}.
$$ 

In terms of $q$, we can rewrite \eqref{eq:result_dimensional} as
\begin{equation}
\frac{C - C_0}{b G_a f} = q \sum_{k=1}^{N}{\sin \left( 2 \pi k q\right)}. \label{eq:result_dimensionless}
\end{equation}

For simplicity's sake, we will hereafter write the LHS as $L \equiv (C - C_0) / b G_a f$. \\

\par
Recall now that we defined $N$ in terms of $\tau$ (and by extension, $q$). So we are not yet done. We have to evaluate the sum to extract $N$, which, after an involved calculation \footnote{The calculation in all its glory is dealt with in \hyperref[subsec:sine_sum]{Appendix A}.}, results in
\begin{equation}
L = q \: \frac{\sin(N\pi q) \sin((N+1)\pi q)}{\sin(\pi q)}. \label{eq:summed_result}
\end{equation}

Substituting 
$$
N = \frac{T}{\tau} = \frac{T}{fq},
$$

we get
\begin{equation}
L = q \: \frac{\sin\left(\frac{T \pi}{f}\right) \sin\left(\frac{T \pi}{f} + \pi q\right)}{\sin(\pi q)}. \label{eq:result_final1}
\end{equation}

To simplify \eqref{eq:result_final1}, define a new dimensionless time $p$ as 
$$
p \equiv \frac{T}{f}.
$$ 

Then, we get our final result,
\begin{equation}
\boxed{L = \frac{C - C_0}{b G_a f} = q \: \frac{\sin(\pi p) \sin(\pi (p+q))}{\sin(\pi q)}}. \label{eq:result_final2}
\end{equation}

\section*{Interpretation}
\label{sec:interpetation}
$L$ is a dimensionless measure of how much we actually pay for gas through time $T$, relative to our ``base" cost of paying the average cost each time. We want to see how that varies with our variables, in this case the dimensionless times $p$ and $q$. \\

\par
We recall $p = T / f$, $q = \tau / f$. Think of this as a ``sampling'' problem, where we are sampling a signal (the gas prices) at periodic intervals of $\tau$, for a length of time $T$. The dimensionless forms of those time metrics are just scaled against a reference time which is the period at which the gas prices fluctuate. In summary, $p$ is a measure of for how long you are doing the analysis. $q$ is a measure of how often you want to fill up gas. $q$ is the critical variable to answer the question in the title of this document. \\

\par
The sinusoids in \eqref{eq:result_final2} make it difficult to predict the behavior, compounded by the presence of three of them in products/ratios. In addition, as $-1 \leq \sin u \leq 1$, the ratio in \eqref{eq:result_final2} could range from $-\infty$ to $+\infty$. Instead, to analyze the result, and to answer the question posed in the title, we must look at some limiting cases.

\section*{Limiting Cases}
\label{sec:limiting_cases}

\subsection*{1) $q \rightarrow 0$, $p$ finite}
In this case,
\begin{align}
\sin (\pi q) &\rightarrow \pi q, \nonumber\\
\sin (\pi(p+q))  &\rightarrow \sin(\pi p), \text{ so} \nonumber\\[0.5em]
L = q \: \frac{\sin(\pi p) \sin(\pi (p+q))}{\sin(\pi q)} &\rightarrow \frac{\sin^2(\pi p)}{\pi}. \label{eq:q_to_zero}
\end{align}

A few remarks:
\begin{itemize}
	\item The result in \eqref{eq:q_to_zero} does not depend on $q$.
	\item Instead of the range from $-\infty < L < +\infty$, $ 0 \leq L \leq 1/\pi$. \\
\end{itemize} 

\par 
Although sensibly unlikely since we cannot fill up gas every moment of our lives, this result provides a hunch that the range of possible losses decreases as ``sampling" rate increases. Just as importantly, it provides a sanity-check for our result in \eqref{eq:result_final2}. \textit{Mathematically} speaking, when $q \rightarrow 0$, our discrete sum that we started setting up in \eqref{eq:sum_C_k} (finite, discrete sampling) approaches an \textit{integral} (continuous sampling). If we replace the sum in \eqref{eq:result_dimensionless} with an integral, the results should match this limiting case, which will affirm the validity of \eqref{eq:result_final2}. Spoiler alert: it indeed does.\footnote{See \hyperref[subsec:sum_to_int]{Appendix B} for the calculation.}


\subsection*{2) $p \gg q$, $q$ finite and nonzero}

 In this case,
\begin{align}
\sin (\pi(p+q))  &\rightarrow \sin(\pi p), \text{ so} \nonumber\\[0.5em]
L = q \: \frac{\sin(\pi p) \sin(\pi (p+q))}{\sin(\pi q)} &\rightarrow \frac{q}{\sin(\pi q)}\sin^2(\pi p). \label{eq:p_to_infty}
\end{align}

This is the most sensible approximation, as it says our sampling time is much longer than the interval of sampling. Which makes sense, if we consider our titular question over a long period of time. Indeed, this may be the only question that matters -- what is our \textit{long-term} strategy? The good news is that the math simplifies in this limiting case. \\

\par
First off, $p$ and $q$ are separable, that is, $L = f(p)g(q)$. Thus, we can set aside the $p$ dependence for now. It is our understanding here that the $\sin^2(\pi p)$ term simply scales $L$ relative to $q$ without changing the fundamental dependence. \\

To begin illustrating my subsequent point, I'm going to now scale $L$ by $\pi$ to define
\begin{equation}
L' \equiv \pi L = f(p) \frac{\pi q}{\sin (\pi q)}.
\end{equation}

Again, as we did before with $\sin^2(\pi p)$, we have just scaled $L$ by a constant relative to $q$, which is 100\% valid. Note that I have replaced the $\sin^2$ term by $f(p)$ to make my earlier point clearer as well. And then to complete the syntactic sugaring, we might as well define $q'$ as $\pi q$:
\begin{equation}
L' = f(p) \frac{q'}{\sin q'}. \label{eq:L_scaled_q_prime}
\end{equation}

Consider the interval $0 < q < 1$, or $0 < q' < \pi$. These are the most common cases, as our sampling period (how often we fill up our gas) $\tau$ is almost always smaller than the period at which the signal (gas prices) fluctuate $f$. (Recall that $q = \tau / f$.) In this interval, we can see that $q'$ will increase faster than $\sin q'$, the latter of which reaches a max at $q' =\pi/2$ and then drops back to zero at $q' = \pi$. Hence the ratio $q'/\sin q'$ is increasing in that interval. So does that mean it's cheaper to fill gas more frequently? Hold up, not yet...

\section*{Phase Shifts}
\label{sec:phase_shifts}
\textit{Intuitively}, we should say whatever result we obtain with \eqref{eq:gas_prices} can be used without loss of generality with any phase shifts to the sine waves. Let us set out to prove or disprove this. Define $\tilde{G}(t)$ such that
\begin{equation}
\tilde{G}(t) = G_0 + G_a \sin \left( \frac{2 \pi (t + t_{\delta})}{f} \right) \label{eq:gas_prices_shifted},
\end{equation} 

where $t_{\delta}$ is an arbitrary horizontal shift to our gas prices ``signal". With this, \eqref{eq:result_dimensional} becomes
\begin{equation}
\frac{C - C_0}{b G_a} = \tau \sum_{k=1}^{N}{\sin \left( \frac{2 \pi (k \tau + t_{\delta})}{f}\right)} \label{eq:result_dimensional_shifted}.
\end{equation}

As in our base calculation, we seek to nondimensionalize the equation, which we will do by defining
$$
s \equiv \frac{t_{\delta}}{f},
$$

in addition to the original $q$, to yield
\begin{equation}
\frac{C - C_0}{b G_a f} = q \sum_{k=1}^{N}{\sin \left( 2 \pi (kq + s) \right)}. \label{eq:result_dimensionless_shifted}
\end{equation}

Evaluating the sum as we did previously, we get\footnote{See \hyperref[subsubsec:sumwith_phase_shift]{Appendix A, Section 2.}}
\begin{equation}
\tilde{L} = q \: \frac{\sin(N\pi q) \: \sin((N+1)\pi q + 2\pi s)}{\sin(\pi q)}, \label{eq:summed_result_shifted}
\end{equation}

where the tilde on the $\tilde{L}$ signifies this is the shifted-sine result. \\

\par Comparing this result to \eqref{eq:summed_result}, we can conclude that the final result involving $p$, $q$, and the newly-minted $s$ will be
\begin{equation}
\boxed{\tilde{L} = \frac{C - C_0}{b G_a f} = q \: \frac{\sin(\pi p) \sin(\pi (p+q) + 2\pi s)}{\sin(\pi q)}}. \label{eq:result_final_shifted}
\end{equation}

Focusing on the \textit{long-term} strategy, we take conditions $p \gg q$:
\begin{equation}
\tilde{L} \xrightarrow{p \gg q}  \frac{q}{\sin(\pi q)} \sin(\pi p) \sin(\pi p + 2\pi s).
\end{equation}

This is once again separable between $p$ and $q$. Scaling $\tilde{L}$ by $\pi$, and then defining our $q' = \pi q$ and $\tilde{f}(p) = \sin(\pi p) \sin(\pi p + 2\pi s)$, we have
\begin{equation}
\tilde{L}' = \tilde{f}(p)\frac{q'}{\sin q'} .  \label{eq:L_scaled_q_prime_shifted}
\end{equation}

Now compare \eqref{eq:L_scaled_q_prime_shifted} to \eqref{eq:L_scaled_q_prime}. The $q'$ dependence is the same. However, $f(p) \neq \tilde{f}(p)$. $f(p) = \sin (\pi p) \geq 0$, whereas $\tilde{f}(p) = \sin (\pi p) \sin (\pi p + 2\pi s)$, such that $-1 \leq \tilde{f}(p) \leq 1$. 

\section*{Upshot} 
Recall that the variables $p$ and $s$ represent how long we're sampling, and the position of the gas prices at $t=0$, respectively. For all intents and purposes, we won't really care about $p$ and $s$ is the ''luck-factor." The magnitude of $L \propto C - C_0$, the deviation from average cost, is always going to be greater the less frequently you sample, for a given $p$ and $s$. When we initially set up the problem, we used $G(t) \sim \sin (t)$, which ended up giving us losses. On closer examination, this should not have been a surprising result, because the gas prices rise initially, giving us a ``bias" of positive values from the get-go. However, if we had replace $\sin(t)$ with $-\sin(t)$, or alternatively using $s = 1/2$ (a half-period shift) in \eqref{eq:result_final_shifted}, the value would have been negative, because a negative ``bias" would have been established in the first quarter-period. $L$ in that case would become more negative as the sampling period increase. Loosely speaking, $q$ controls the magnitude of $L$, while $p$ and $s$ the sign (loss is $L > 0$, gain is $L < 0$).\\

\par
With regards to $q$, all this mathematical work has shown that we can \textbf{minimize the maximum loss} by minimizing $q$. However, the maximum amount you could stand to gain also decreases with $q$. Though this analysis was done with gas prices, we can use this result with, say, stocks. Instead of sampling for some long time, we want to invest in a certain amount of stocks. They also oscillate with some frequency. The common knowledge in the investment world is to invest in small bundles of the stock frequently, rather than to purchase a lump-sum of it. This is known as \href{https://en.wikipedia.org/wiki/Dollar_cost_averaging}{dollar-cost averaging}. Frequent purchasing of small units hedges against bad luck, and larger losses.\\

\par 
So to answer the question in the title: \textbf{I want to minimize my losses, and therefore, by the math done in this article, I will fill up more frequently}.


\section*{Appendix}

\subsection*{A. Finite sum of sines}
\label{subsec:sine_sum}

\renewcommand{\theequation}{A.\arabic{equation}}
\renewcommand{\Im}[1]{\text{Im}\left({#1}\right)}
\renewcommand{\exp}[1]{e^{#1}}
In this appendix we evaluate the sum in \eqref{eq:result_dimensionless}. Let $S$ be the value of the sum:
\begin{equation}
S = \sum_{k=1}^{N} \sin (2 \pi k q).
\end{equation}

Here we opt to use complex exponentials, realizing that
\begin{equation}
\sum_{k=1}^{N} \sin (2 \pi k q) = \sum_{k=1}^{N} \Im{\exp{i(2 \pi k q)}} = \Im{\sum_{k=1}^{N} \exp{i(2 \pi k q)}}.
\end{equation}

Defining
$$
r \equiv e^{i(2 \pi q)},
$$

we get
\begin{equation}
S = \Im{\sum_{k=1}^{N} r^{k}}.
\end{equation}

This is a geometric series, which can be easily summed explicitly:
\begin{equation}
\sum_{k=1}^{N} r^{k} = \frac{r - r^{N+1}}{1 - r}
\end{equation}

and hence, plus followed by some algebraic manipulation
\begin{align}
S = \Im{\frac{r - r^{N+1}}{1 - r}} &= \Im{\frac{\exp{i 2\pi q} - \exp{i 2\pi q (N+1)}}{1 - \exp{i 2\pi q}}} \nonumber \\[0.4em]
 &= \Im{\exp{i 2\pi q} \: \frac{1 - \exp{i 2\pi Nq}}{1 - \exp{i 2\pi q}}}. \label{eq:cmplxexp_ratio}
\end{align}

To continue we must manipulate the ratio expression inside \eqref{eq:cmplxexp_ratio},
$$
\frac{1 - \exp{i 2\pi Nq}}{1 - \exp{i 2 \pi q}} = 
\frac{\cancel{2i} \: \exp{i \pi Nq}\left(\frac{\exp{-i\pi Nq} - \exp{i\pi Nq}}{2i}\right)}{\cancel{2i} \: \exp{i\pi q} \left( \frac{\exp{-i\pi q} - \exp{i\pi q}}{2i}\right)}.
$$

Of course we do this because we realize
$$
\sin u = \frac{\exp{iu} - \exp{-iu}}{2i}.
$$

So 
$$
\frac{1 - \exp{i 2\pi Nq}}{1 - \exp{i 2 \pi q}} = \frac{\exp{i\pi Nq}\sin(N\pi q)}{\exp{i\pi q}\sin(\pi q)}
$$

and
\begin{align}
S = \Im{\exp{i 2\pi q} \: \frac{1 - \exp{i 2\pi Nq}}{1 - \exp{i 2\pi q}}} 
&= \Im{\exp{i 2\pi q} \: \frac{\exp{i\pi Nq}\sin(N\pi q)}{\exp{i\pi q}\sin(\pi q)}} \nonumber \\[0.5em]
&= \Im{\exp{i \pi (N+1)q} \: \frac{\sin(N\pi q)}{\sin(\pi q)}} \label{eq:after_long_summation}.
\end{align}

But the sines are real, so we can pull them out of the imaginary part:
\begin{align}
S &= \frac{\sin(N\pi q)}{\sin(\pi q)} \: \Im{\exp{i \pi (N+1)q}} \\[0.5em]
&= \frac{\sin(N\pi q)}{\sin(\pi q)} \: \sin((N+1)\pi q),
\end{align}

where in the last step we actually take the imaginary part, undoing what we did in the first step. $qS$ is just the result in \eqref{eq:summed_result}, so we're done here. QED.

\subsubsection*{Adding the phase shift}
\label{subsubsec:sumwith_phase_shift}
Now we extend this result to account for the phase shift, as in \eqref{eq:result_dimensionless_shifted}. Let
\begin{equation}
\tilde{S} = \sum_{k=1}^{N} \sin(2\pi (kq + s)).
\end{equation}

We use the complex exponential trick again:
\begin{equation}
\tilde{S} = \Im{\sum_{k=1}^{N} \exp{i(2\pi (kq + s))}} = \Im{\exp{i(2\pi s)}\sum_{k=1}^{N} \exp{i(2\pi kq)}} \label{eq:step1_shifted},
\end{equation}

where in the latter equality we have taken $\exp{i(2\pi s)}$ out of the summation as there is no $k$-dependence. \\

\par The latter sum was just done in the previous subsection of Appendix A, so we substitute the result in \eqref{eq:after_long_summation} in:
\begin{align}
\tilde{S} = \Im{\exp{i(2\pi s)}\sum_{k=1}^{N} \exp{i(2\pi kq)}} &= \Im{\exp{i \pi (N+1)q} \: \exp{i 2\pi s} \: \frac{\sin(N\pi q)}{\sin(\pi q)}} \nonumber \\[0.5em]
&= \frac{\sin(N\pi q)}{\sin(\pi q)} \: \Im{\exp{i (\pi (N+1)q + 2\pi s)}} \\[0.5em]
&= \frac{\sin(N\pi q)}{\sin(\pi q)} \: \sin ((N+1)\pi q + 2\pi s), \label{eq:step2_shifted}
\end{align}

following algebraic steps similar to the last subsection in our work between \eqref{eq:step1_shifted} and \eqref{eq:step2_shifted}. In similar fashion $q\tilde{S}$ equals the result in \eqref{eq:summed_result_shifted}. QED.

\subsection*{B. Sum of sines to integral}
\label{subsec:sum_to_int}

\renewcommand{\theequation}{B.\arabic{equation}}
Again we are at our favorite point,
\begin{equation}
S = \sum_{k=1}^{N} \sin (2 \pi k q).
\end{equation}

Now we let $q \rightarrow 0$. However, $p$ must remain fixed which means $N$ becomes very large. If we think about the original discrete sum, we sum our sine function at unit intervals of $k$. However, since $N$ becomes large now, $k$ looks very small. Thus, our sum along $k$ can be approximated by an integral along $k$:
\begin{equation}
\sum_{k=1}^{N} \sin (2 \pi k q) \rightarrow \int_0^N{\sin(2\pi k q) \: dk}.
\end{equation}

This is straightforward integral to evaluate.
\begin{align}
\int_0^N{\sin(2\pi k q) \: dk} &= -\left.\frac{\cos(2\pi k q)}{2\pi q}\right\vert_{0}^{N}  \\[0.5em]
&= \frac{1}{\pi q} \left(\frac{1 - \cos(2\pi q N)}{2}\right) \nonumber \\[0.5em]
&= \frac{\sin^2(\pi q N)}{\pi q},
\end{align}

where we have used the identity $\cos 2u = 1 - 2\sin^2 u$ in the last step. \\

\par Finally, recall
$$
N = \frac{T}{fq} \Rightarrow Nq = \frac{T}{f} = p.
$$

Hence
\begin{equation}
S \xrightarrow{q \rightarrow 0} \frac{\sin^2(\pi q N)}{\pi q} = \frac{\sin^2(\pi p)}{\pi q}.
\end{equation}

And $qS$ results in the expression in \eqref{eq:q_to_zero}, which is what we expected all along. QED.

\end{document}